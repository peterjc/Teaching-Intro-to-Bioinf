%% UNCOMMENT FOR SLIDES
\input{sections/preamble_slides}
%% UNCOMMENT FOR HANDOUTS
%\input{sections/preamble_handouts}
%% GENERIC STYLE SETTINGS BELOW
\input{sections/preamble_style}
% LISTINGS SETTING
\input{sections/preamble_codelistings}

%%%
% TITLE
\title[Intro to Bioinformatics] % (optional, only for long titles)
{An Introduction to Bioinformatics Tools}
\subtitle{Part 1: Golden Rules of Bioinformatics}
\author[Pritchard, Cock] % (optional, for multiple authors)
{Leighton~Pritchard and Peter~Cock}
\institute[The James Hutton Institute] % (optional)
{
  Information and Computational Sciences\\
  The James Hutton Institute
}
\date[May 2014] % (optional)
{Bioinformatics Training, 29$^{th}$, 30$^{th}$ May 2014}
\subject{Bioinformatics}

%%%
% TOC
\input{sections/preamble_toc}

%%%
% START DOCUMENT
\begin{document}

\frame[plain]{\titlepage}
  
% Charles Darwin quote
%\include{sections/slide_darwinquote}
   
    
 %%%
 % SECTION: Golden Rule 0
\section{Rule 0}
  \include{sections/subsection_goldenrule0}

 %%%
 % SECTION: Golden Rule 1
\section{Rule 1}
%\include{sections/subsection_goldenrule1exercise}
\subsection{Golden Rule 1 Exercise}
\begin{frame}[fragile]
  \frametitle{Exercise 1 (Abridged)}
  \framesubtitle{Summary of commands in the slides}
  \begin{itemize}
    \item You will use \texttt{R} at the command-line to analyse your data
  \end{itemize}
\begin{lstlisting}[language=bash]
$ cd ~/introToBioinformatics/data/ex1_expression/
$ R
\end{lstlisting}

\end{frame}
\begin{frame}[fragile]
  \frametitle{Exercise 1 (Abridged)}
  \framesubtitle{Summary of commands in the slides}
  \begin{itemize}
    \item You are in group A, B, C or D - this decides your dataset\\
    \texttt{expnA.tab}, \texttt{expnB.tab}, \texttt{expnC.tab}, or \texttt{expnD.tab}
  \end{itemize}
\begin{lstlisting}[language=R]
> data = read.table("expnA.tab", sep="\t", header=TRUE)
> head(data)
  gene1 gene2
1    10  8.04
2     8  6.95
3    13  7.58
4     9  8.81
5    11  8.33
6    14  9.96
> mean(data$gene1)
[1] 9
> mean(data$gene2)
[1] 7.500909
> sd(data$gene1)
[1] 3.316625
> sd(data$gene2)
[1] 2.031568
> cor(data)
          gene1     gene2
gene1 1.0000000 0.8164205
gene2 0.8164205 1.0000000
\end{lstlisting}
\end{frame}

  \include{sections/subsection_goldenrule1}  

 %%%
 % SECTION: Golden Rule 2
\section{Rule 2}
%\include{sections/subsection_goldenrule2exercise}
\subsection{Golden Rule 2 Exercise}
\begin{frame}[fragile]
  \frametitle{Exercise 2 (Abridged)}
  \framesubtitle{Summary of commands in the slides}
  \begin{itemize}
    \item You are in group A, B, C or D - this decides your database\\
    \texttt{dbA}, \texttt{dbB}, \texttt{dbC}, or \texttt{dbD}
  \end{itemize}
\begin{lstlisting}[language=bash]
$ cd ../ex2_blast
$ script
Script started, output file is typescript
$ blastp -num_alignments 1 -num_descriptions 1 -query query.fasta -db dbA
...
$ exit
Script done, output file is typescript
\end{lstlisting}
\begin{lstlisting}[language=bash]
$ cat typescript
Script started on ...
\end{lstlisting}
\end{frame}

  \include{sections/subsection_goldenrule2}  

 %%%
 % SECTION: Golden Rule 3
\section{Rule 3}
%\include{sections/subsection_goldenrule3exercise}
\begin{frame}
  \frametitle{Exercise 3 (Abridged)}
  \framesubtitle{Classification}
  \begin{itemize}
    \item Rule: If there is a vowel on one side of a card, there \textit{must} be an even number on the other side of that card.
    \item Is this rule true?
    \item Which cards, when turned over, can help determine if this rule is true? (How many are there?)
  \end{itemize}
  \includegraphics[width=0.8\textwidth]{images/wason}
\end{frame}

  \include{sections/subsection_goldenrule3}  

%%%
% SECTION: Conclusion
\section{Conclusions}
  \include{sections/subsection_goldenruleconclusions}  

% etc
\end{document}